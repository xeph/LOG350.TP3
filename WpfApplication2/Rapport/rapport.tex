%\documentclass[12pt,a4paper,oneside]{reportu}
\documentclass[letterpaper, oneside, 12pt, these, creativecommons]{thETS}
\usepackage{times}
\usepackage[pdftex]{graphicx}
\usepackage{multirow}
\usepackage{pdfpages}
\usepackage{amsmath}
\usepackage{amssymb}
\usepackage{caption}
\usepackage[utf8]{inputenc}
\usepackage[frenchb]{babel}
\usepackage{amsmath}
\usepackage{caption}
\usepackage{multibib}
\usepackage{url}
\urlstyle{rm}

\begin{document}

\includepdf[pages=-]{pageTitre.pdf}

\tableofcontents
\listoftables
\listoffigures

\chapter{Glossaire}

\begin{description}
\item[C\#] est un langage de programmation orienté objet à typage fort, créé par la société Microsoft. \footnote{\url{https://fr.wikipedia.org/wiki/C_sharp}}
\item[Windows Presentation Foundation] est la spécification graphique de Microsoft .NET 3.0. Il intègre le langage descriptif XAML qui permet de l'utiliser d'une manière proche d'une page HTML pour les développeurs. \footnote{\url{https://fr.wikipedia.org/wiki/Windows_Presentation_Foundation}}
\item[WinForms] est le nom de l'interface graphique qui est incluse dans .NET Framework, fournissant l'accès via du Managed code à l'API Windows. \footnote{\url{https://fr.wikipedia.org/wiki/Windows_Forms}}
\item[framework] est un kit de composants logiciels structurels, qui sert à créer les fondations ainsi que les grandes lignes de tout ou d’une partie d'un logiciel (architecture). \footnote{\url{https://fr.wikipedia.org/wiki/Framework}}
\item[.NET 4.5] est un framework pouvant être utilisé par un système d'exploitation Microsoft Windows et Microsoft Windows Mobile depuis la version 5 (.NET Compact Framework). \footnote{\url{https://fr.wikipedia.org/wiki/Framework_.NET}}
\item[SQLite] est une bibliothèque écrite en C qui propose un moteur de base de données relationnelle accessible par le langage SQL.\footnote{\url{https://fr.wikipedia.org/wiki/SQLite}}
\end{description}

\chapter{Introduction et sommaire du travail effectue en TP2}

L'application permet la gestion simple de taches et d'evenements tout en offrant des fonctionnalites de gestion et de classification avancees. Parmi celles-ci, notons la possibilite de definir des rappels, de categoriser les elements, de definir un evenement comme echeance d'une tache ou encore de definir des sous-taches a une tache imposante.

Durant notre analyse de tache, nous avons determiner que le publique cible est des personnes de 15 ans et plus desirant organiser son temps et sa vie professionnelle. De plus, nous avons conclu que notre interface devrais ressembler a des interfaces connu pour la garder familiere et garder une courbe d'apprentissage faible. Pour ce faire, nous avons dresser une liste des cas d'utilisation que l'application doit traiter pour se donner un point de depart. Par la suite, nous avons construit nos prototypes statiques.

Pour construire notre prototype statique, nosu avons utiliser des fenetres virtuelles ou chaques points de chaque fenetre a ete detaille pour s'assurer que le travail devant etre effectue par chaque interface a bel et bien ete compris. De plus, chaque fonctionnalite a aussi ete detaille dans le but qu'elles soient bien comprise.

Jusqu'a present, aucune modification n'a ete apporte entre les prototypes statiques et dynamiques. Tout va rester comme indique dans le document precedent.

Dans le present document, les tests qui seront effectue avec des utilisateurs seront listes et detailles. Par la suite, un concensus sera fais pour chaque tache et une amelioration qui pourrait etre faite sera proposer pour ameliorer le comportement du logiciel.

\chapter{Planification du travail}

\begin{table}
	\centering
	\begin{tabular}{|l|l|}
		\hline
		Semaine	& Travail accomplis 								\\ \hline
		4 novembre	& Choix de la technologie et apprentissage de celle-ci.		\\ \hline
		11 novembre	& Marc-André : Conception des interfaces. 				\\
				& Martin : Conception de la base de données.				\\
				& Simon : Conception de la base de données.				\\ \hline
		18 novembre	& Marc-André : Début de la rédaction du rapport. 			\\ 
				& Martin : Programmation du prototype dynamique. 			\\
				& Simon : Programmation du prototype dynamique. 			\\ \hline
		25 novembre	& Marc-André : Rédaction du rapport.		 			\\
				& Martin : Programmation du prototype dynamique. 			\\ 
				& Simon : Programmation du prototype dynamique.		 	\\ \hline
		2 decembre	& Marc-André : Rencontre avec les utilisateurs. 			\\
				& Martin : Rédaction du rapport.						\\
				& Simon : Rédaction du rapport.						\\ \hline
		9 decembre	& Marc-André : Préparation de la présentation.	 		\\ 
				& Martin : Préparation de la présentation.	 			\\ 
				& Simon : Préparation de la présentation.	 			\\ \hline
	\end{tabular}
	\caption{Echéancier}
\end{table}

\chapter{Realisation du prototype dynamique}

Dans la section suivante, nous vous présenterons notre prototype dynamique et donc, avant d'entreprendre la lecture de cette section, nous vous conseillons de lire le document se trouvant a l'annexe III. Ce document contient les informations concernant le prototype statique et toute la logique derriere le prototype dynamique.

\section{Choix des outils}

Pour realiser le prototype dynamique, nous avons decide d'utiliser le C\# et le Windows Presentation Foundation (WPF) comme langages de programmation. Nous avons choisi ces langages tout simplement parce que ceci nous permet de séparer le code des interfaces et que le WPF va ultimement remplacer WinForms. 

Comme environnement de developpement, nous utilisons Visual Studio 2012 avec le framework .NET 4.5 pour faciliter et accelerer notre developpement. Ce framework nous donne des composantes graphiques de base qui nous simplifie la vie en nous permettant d'economiser du temps pour ne pas a avoir a developper des composants de base comme un TextBox ou un Label, par exemple.

De plus, comme base de donnees, nous utilisons SQLite parce que cela nous permet d'avoir un endroit pour stocker nos donnees de facon structure et le tout dans un seul et unique fichier.

Pour assurer la gestion des versions, nous utilisons un logiciel connu sous le nom de Git. Ce logiciel s'occupe de faire la gestion des versions de chaque fichier du projet.

Pour la production de la documentation, nous utilisons le processeur de texte \LaTeX.

\newpage

\section{Captures d'ecran}

\subsection{Fenetre principale}

\newpage

\subsection{Evenement}

\newpage

\subsection{Tache}

C'est a partir de cet ecran que l'utilisateur peut modifier ou creer une tache. A partir de cet ecran, l'utilisateur peut aussi faire la gestion des alertes et des sous-taches appartenant a la tache parent. A la page 19 du document a l'annexe III, il y a de plus ample informations concernant cette fenetre.

\begin{figure}[H!]
	\centering
	\includegraphics[width=1\textwidth]{fenetre_tache.png}
	\caption{Fenetre concernant une tache.}
\end{figure}

\newpage

\subsection{Priorites}

C'est a partir de cet ecran que l'utilisateur peut faire la gestion des diverses priorites disponibles dans l'application. A la page 22 du document a l'annexe III, il y a de plus ample informations concernant cette fenetre.

\begin{figure}[H!]
	\centering
	\includegraphics[width=1\textwidth]{fenetre_priorite.png}
	\caption{Fenetre concernant les priorites.}
\end{figure}

\newpage

\section{Justification des choix de conception}

Divers patrons de conception ont ete utilise pour parvenir a la conception de l'application que nous avons presentement. 

Parmi les divers patrons disponibles, on compte le patron \emph{Many Workspaces} qui se retrouve a la fenetre principale de l'application ou l'utilisateur a la possibilite d'avoir plusieurs listes d'ouvertes et de les organiser comme il veut via des onglets. L'utilisateur peut donc gerer plusieurs listes a la fois. 

Pour la navigation entre les fenetres de l'application nous utilisons le patron \emph{Escape Hatch}. Ainsi, la navigation entre les fenetres est limite et l'utilisateur peut toujours revenir a la fenetre principale sans chercher pendant de longues minutes comment y revenir. Cela simplifie grandement l'interaction entre l'application et l'utilisateur en simplifiant le processus de navigation au maximum. 

Ensuite, pour ce qui est des listes, nous utilisons le patron \emph{Tree Table} sur, par exemple, la fenetre principale. Avec ce patron, nous listons donc les taches et chaque sous-tache sous sa tache correspondante sous forme d'arborescence. Nous pouvons donc afficher un maximum d'information utile a l'utilisateur lorsque celui-ci le demande. Nous utilisons aussi le patron \emph{New-Item Row} dans la fenetre Priorites pour permettre l'ajout rapide et infini de lignes dans le tableau sans que l'utilisateur n'ait besoin d'appuyer sur quoi que se soit. Pour ce qui est de toutes les grilles, le patron \emph{Sortable Table} s'applique et permet de trier l'information affiche en tout temps pour permettre a l'utilisateur de trouver ce qu'il veut plus rapidement. 

De plus, nous avons respecte le plus possible le contenu de la norme \emph{ISO 9241-120} a \emph{ISO 9241-129} qui traite sur les interactions sur les entrees et sorties.

Quelques changements on ete effectue entre le prototype statique et dynamique. Parmi ces changements, on compte les options possibles pour une echeance dans la fenetre sur les taches. Ce changement a ete fait a cause de limitations logiciels et par faute de temps. Il est a noter que ces changements n'influence en aucun cas l'efficacite de l'application.

\section{Ameliorations possibles}

Quelques ameliorations possibles seraient d'utiliser les lois psychomotrices de Fitts et Miller pour optimiser l'interface. Nous pourrions ainsi optimiser les deplacements que l'utilisateur doit faire avec sa souris pour cliquer sur les boutons et les champs de saisies dans les fenetres. Mais, par faute de temps et de ressources humaines, il nous est donc impossible d'effectuer ces tests.

\chapter{Demonstration du prototype dynamique en laboratoire}

Un prototype dynamique de l'application est disponible a l'adresse suivante : \\
\url{https://github.com/xeph/LOG350.TP3}.

évaluation du prototype en laboratoire  : le 4 décembre 2012 au laboratoire (pour les 2 groupes).

exposé oral  : le 6 décembre 2012 dans un local à définir (pour les 2 groupes).

remise du code final : le 11 décembre 2012 à 23h59 maximum par courriel à log350a2012@gmail.com.

remise du rapport :  le 11 décembre 2012 à 23h59 maximum en version papier dans la chute du département.

remise de la présentation de l’éxposé oral : le 6 décembre 2012 à 23h59 maximum par courriel à log350a2012@gmail.com .

\chapter{Tests avec utilisateurs}

\section{Methodologie}

Avant d'effectuer les tests, un document expliquant la nature du projet et le deroulement des tests lui a ete fourni. Les membres de l'equipe assument que le document a ete lu lors de la rencontre. De plus, un formulaire de consentement sera obligatoirement signe par l'utilisateur et un membre de l'equipe pour que les deux parties s'entendent sur ce qui sera fait lors de la rencontre.

Pour faire les tests avec les utilisateurs, le ou les membres de l'equipe s'asseoiront avec l'utilisateur qui testera l'application. Par la suite, un membre de l'equipe se chargera de dire chaque tache que l'utilisateur devra effectuer, et si l'utilisateur a des questions, il sera charge de lui fournir les informations. Lorsque l'utilisateur aura terminer de faire une tache, il y aura une breve pause pour que tout le monde ait le temps de prendre des notes. L'execution de la tache suivante se fera lorsque tout le monde aura terminer de prendre les notes pour la tache courante. Quand toutes les taches auront ete execute, l'utilisateur pourra donner son opinion et ses impressions sur l'application sur des choses a ameliorer, les points positifs et des fonctionnalitees qui pourraient etre interessante a ajouter.

Les tests se feront dans le calme et le respect. De plus, l'utilisateur peut, en tout temps, decider d'arreter les tests sans devoir fournir une raison. Les donnees recueillis seront cependant conserve a moins que l'utilisateur ne demande a ce qu'elles soient detruites.

\newpage

\section{Liste des taches}

\begin{table}
	\centering
	\begin{tabular}{|l|l|l|}
		\hline
		no Tache	& Titre et description		& Elements que vous voulez verifier et hypotheses 	\\ \hline 
		1		& Ajouter un evenement		&  							\\ 
				& sans alertes.			&							\\ \hline
		2		& Modifier l'evenement creer et	&							\\
				& ajouter une alerte.		&							\\ \hline
		3		& Ajouter le tag ecole		&							\\
				& a l'evenement.			&							\\ \hline
		4		& Creer un evenement et		&							\\
				& ajouter le tag travail.		&							\\ \hline
		5		& Rechercher les evenements	& 							\\
				& par tag.				&							\\ \hline
		6		& Supprimer les evenements	&							\\
				& contenant le tag travail.		&							\\ \hline
		7		& Trouver la fenetre des		&							\\
				& priorites.				& 							\\ \hline
		8		& Ajouter une nouvelle priorite 	&							\\
				& valide.				&							\\ \hline
		9		& Desactiver une priorite ne	&							\\
				& contenant aucun evenement ou &							\\
				& tache associe.			&							\\ \hline
		10		& Ajouter plusieurs taches		&							\\
				& (3-5 environ).			&							\\ \hline
		11		& Prendre une tache et lui		&							\\
				& affecter 2 sous-taches.		&							\\ \hline
		12		& Modifier la completion d'une	&							\\
				& tache non complete.		&							\\ \hline
		13		& Modifier la completion de	&							\\
				& sous-taches pour completer	&							\\ 
				& une tache.				&							\\ \hline
		14		& Supprimer une tache complete.	&							\\ \hline
		15		& Modifier les informations		&							\\
				& d'une tache non complete.	&							\\ \hline
		16		& Rechercher une tache avec la 	&							\\
				& priorite cree precedement.	& 							\\ \hline
	\end{tabular}
	\caption{Liste des taches}
\end{table}

\section{Liste des utilisateurs}

Les caracteristiques principales des utilisateurs choisis sont la gestion de leurs taches pour les travaux et devoirs au CEGEP ou bien a l'universite et la gestion des rendez-vous pour les professionnels. Nous avons choisis ce nombre d'utilisateur, car il nous fallait un petit echantillon oeuvrant dans le meme type de vie que le publique visee mais dans des spheres professionnelles differentes. Il est pertinant d'avoir teste avec ces utilisateurs parce que se sera principalement ce type d'utilisateur qui se servira d'une application comme celle-ci.

\newpage

\section{Resultats}

\begin{table}
	\centering
	\begin{tabular}{|l|l|l|}
	\hline
	no Tache	& Points importants de l'observation	\\ \hline
	1		& 0						\\ \hline
	2		& 0						\\ \hline
	3		& 0						\\ \hline
	4		& 0						\\ \hline
	5		& 0						\\ \hline
	6		& 0						\\ \hline
	7		& 0						\\ \hline
	8		& 0						\\ \hline
	9		& 0						\\ \hline
	10		& 0						\\ \hline
	11		& 0						\\ \hline
	12		& 0						\\ \hline
	13		& 0						\\ \hline
	14		& 0						\\ \hline
	15		& 0						\\ \hline
	16		& 0						\\ \hline
	\end{tabular}
	\caption{Resultats de l'utilisateur 1}
\end{table}

\newpage

\begin{table}
	\centering
	\begin{tabular}{|l|l|l|}
	\hline
	no Tache	& Points importants de l'observation	\\ \hline
	1		& 0						\\ \hline
	2		& 0						\\ \hline
	3		& 0						\\ \hline
	4		& 0						\\ \hline
	5		& 0						\\ \hline
	6		& 0						\\ \hline
	7		& 0						\\ \hline
	8		& 0						\\ \hline
	9		& 0						\\ \hline
	10		& 0						\\ \hline
	11		& 0						\\ \hline
	12		& 0						\\ \hline
	13		& 0						\\ \hline
	14		& 0						\\ \hline
	15		& 0						\\ \hline
	16		& 0						\\ \hline
	\end{tabular}
	\caption{Resultats de l'utilisateur 2}
\end{table}

\newpage

\begin{table}
	\centering
	\begin{tabular}{|l|l|l|}
	\hline
	no Tache	& Points importants de l'observation	\\ \hline
	1		& 0						\\ \hline
	2		& 0						\\ \hline
	3		& 0						\\ \hline
	4		& 0						\\ \hline
	5		& 0						\\ \hline
	6		& 0						\\ \hline
	7		& 0						\\ \hline
	8		& 0						\\ \hline
	9		& 0						\\ \hline
	10		& 0						\\ \hline
	11		& 0						\\ \hline
	12		& 0						\\ \hline
	13		& 0						\\ \hline
	14		& 0						\\ \hline
	15		& 0						\\ \hline
	16		& 0						\\ \hline
	\end{tabular}
	\caption{Resultats de l'utilisateur 3}
\end{table}

\newpage

\section{Discussion et recommandations}

\begin{table}
	\centering
	\begin{tabular}{|l|l|l|}
	\hline
	no Tache	& Resume	& Recommandation 	\\ \hline
	1		& 0		& 0 			\\ \hline
	2		& 0		& 0 			\\ \hline
	3		& 0		& 0 			\\ \hline
	4		& 0		& 0 			\\ \hline
	5		& 0		& 0 			\\ \hline
	6		& 0		& 0 			\\ \hline
	7		& 0		& 0 			\\ \hline
	8		& 0		& 0 			\\ \hline
	9		& 0		& 0 			\\ \hline
	10		& 0		& 0 			\\ \hline
	11		& 0		& 0 			\\ \hline
	12		& 0		& 0 			\\ \hline
	13		& 0		& 0 			\\ \hline
	14		& 0		& 0 			\\ \hline
	15		& 0		& 0 			\\ \hline
	16		& 0		& 0 			\\ \hline
	\end{tabular}
	\caption{Recommandations par les observateurs}
\end{table}

\begin{table}
	\centering
	\begin{tabular}{|l|l|l|}
	\hline
	no Tache	& Resume	& Recommandation 	\\ \hline
	1		& 0		& 0 			\\ \hline
	2		& 0		& 0 			\\ \hline
	3		& 0		& 0 			\\ \hline
	4		& 0		& 0 			\\ \hline
	5		& 0		& 0 			\\ \hline
	6		& 0		& 0 			\\ \hline
	7		& 0		& 0 			\\ \hline
	8		& 0		& 0 			\\ \hline
	9		& 0		& 0 			\\ \hline
	10		& 0		& 0 			\\ \hline
	11		& 0		& 0 			\\ \hline
	12		& 0		& 0 			\\ \hline
	13		& 0		& 0 			\\ \hline
	14		& 0		& 0 			\\ \hline
	15		& 0		& 0 			\\ \hline
	16		& 0		& 0 			\\ \hline
	\end{tabular}
	\caption{Recommandations par les utilisateurs}
\end{table}

\chapter{Changements recommandes}

\chapter{Conclusion}

\appendix
\multiannexe

\chapter{Formulaire de consentement des utilisateurs}

\includepdf[pages=-]{form.pdf}

\chapter{Document de présentation du projet}

\includepdf[pages=-]{document.pdf}

\chapter{Travail Pratique 2}

\includepdf[pages=-]{T2.pdf}

\end{document}
