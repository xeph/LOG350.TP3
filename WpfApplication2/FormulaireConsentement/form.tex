\documentclass[12pt,a4paper,oneside]{article}
\usepackage{times}
\usepackage[pdftex]{graphicx}
\usepackage{multirow}
\usepackage{pdfpages}
\usepackage{amsmath}
\usepackage{amssymb}
\usepackage{caption}
\usepackage[utf8]{inputenc}
\usepackage[frenchb]{babel}
\usepackage{amsmath}
\usepackage{caption}
\usepackage{multibib}
\usepackage{url}
\urlstyle{rm}

\begin{document}

\title{Formulaire de consentement}
\date{\today}
\maketitle

Avant d’accepter de participer à ce projet, veuillez prendre le temps de lire et de comprendre les renseignements qui suivent. Ce document vous explique le but de ce projet, ses procédures, avantages, risques et inconvénients. Nous vous invitons à poser toutes les questions que vous jugerez utiles à la personne qui vous présente ce document. \footnote{\url{http://www.cerul.ulaval.ca/doc/Guide_rediger_formulaire_consentement.pdf}}

Cette étude consiste à recueillir des informations concernant votre psychomotricité, la performance de l'application et l'usabilité de l'application. C'est avec ces information que nous pourrons déterminer les améliorations que nous devons apporter à cette application pour que celle-ci rencontre les standards de psychomotricité, de performance et d'usabilité.

Lors des tests, un évaluateur s'asseoira avec vous et vous dictera les actions que vous devrez effectuer. Pendant que vous effecturez ces actions, celui-ci prendra des notes sur votre facon de les effectuer et de tout problème rencontré. De plus, il prendra en note tout commentaire dont vous lui ferez part. Le temps requis pour effectuer les tests est de 1 à 2 heures, selon la rapidité de chaque utilisateur.

Il n'y a aucun avantages, risques ou inconvénients possibles liés à la participation à ce projet.

Vous pouvez en tout temps refuser de participer à se projet ou mettre fin en tout temps à votre participation et toutes les données vous concernant seront détruites. Vous pouvez aussi refuser de répondre à certaines questions sans conséquence.

Les données seront gardées de facon confidentielle, il n'y aura aucune information d'inscrites sur les données recueillies qui permettront de vous identifier. Il sera donc impossible de vous associer aux données recueillies. Les données seront conservées pour une durée indéterminée et ne seront accessible qu'aux personnes de l'équipe, chargés de laboratoire et professeur. De plus, les données seront conservées sur format numérique (pdf).

\newpage

Je soussigné(e) \underline{\hspace{4.5cm}} consens librement à participer au projet intitulée : « (titre complet de la recherche) ». J’ai pris connaissance du formulaire et je comprends le but, la nature, les avantages, les risques et les inconvénients du projet de recherche. Je suis satisfait(e) des explications, précisions et réponses que les étudiants m’ont fournies, le cas échéant, quant à ma participation à ce projet. \footnote{\url{http://www.cerul.ulaval.ca/doc/Guide_rediger_formulaire_consentement.pdf}}

Votre collaboration est précieuse pour la réalisation de notre projet et nous vous remercions d'y participer.

\vspace{1cm}

Utilisateur

\vspace{0.5cm}

Nom  \underline{\hspace{4.5cm}}
Signature  \underline{\hspace{4.5cm}}

\vspace{1cm}

Evaluateur

\vspace{0.5cm}

Nom  \underline{\hspace{4.5cm}}
Signature  \underline{\hspace{4.5cm}}

\end{document}